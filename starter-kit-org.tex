% Created 2011-11-23 Wed 00:21
\documentclass[11pt]{article}
\usepackage[utf8]{inputenc}
\usepackage[T1]{fontenc}
\usepackage{fixltx2e}
\usepackage{graphicx}
\usepackage{longtable}
\usepackage{float}
\usepackage{wrapfig}
\usepackage{soul}
\usepackage{textcomp}
\usepackage{marvosym}
\usepackage[integrals]{wasysym}
\usepackage{latexsym}
\usepackage{amssymb}
\usepackage{hyperref}
\tolerance=1000
\usepackage{amsmath}
\providecommand{\alert}[1]{\textbf{#1}}
\begin{document}



\title{Starter Kit Org}
\author{Chris Beard}
\date{23 November 2011}
\maketitle


This is part of the \href{file:///Users/FingerMan/.emacs.d/starter-kit.org}{Emacs Starter Kit}.

\section*{Starter Kit Org}
\label{sec-1}

Configuration for the eminently useful \href{http://orgmode.org/}{Org Mode}.

Org-mode is for keeping notes, maintaining ToDo lists, doing project
planning, and authoring with a fast and effective plain-text system.
Org Mode can be used as a very simple folding outliner or as a complex
GTD system or tool for reproducible research and literate programming.

For more information on org-mode check out \href{http://orgmode.org/worg/}{worg}, a large Org-mode wiki
which is also \textbf{implemented using} Org-mode and \href{http://git-scm.com/}{git}.
\subsection*{My Latex stuff}
\label{sec-1_1}

\begin{verbatim}
(setq org-export-copy-to-kill-ring nil) ;so it doesn't fill up the clipboard

    ;;; Great help from here
    ;;; http://www.mail-archive.com/emacs-orgmode@gnu.org/msg07645.html
(require 'org-install)
(require 'org-exp)
(require 'org-latex)

  ;;; Let it treat $$ as parentheses
(modify-syntax-entry ?$ "$" org-mode-syntax-table)
(modify-syntax-entry ?\\ "w" org-mode-syntax-table)
(modify-syntax-entry ?& "-" org-mode-syntax-table)
(modify-syntax-entry ?= "-" org-mode-syntax-table)
(modify-syntax-entry ?\ "-" org-mode-syntax-table)

;; (require 'org-atom)
;; (require 'org-exp-bibtex)


(eval-after-load 'org-export-latex
  '(progn
     ;; http://permalink.gmane.org/gmane.emacs.orgmode/36716
     (add-to-list 'org-export-latex-packages-alist '("" "amsmath" t))
     (setcar (rassoc '("wasysym" t) org-export-latex-default-packages-alist)
             "integrals")

     (add-hook 'org-mode-hook
               (lambda ()
                 (setq org-export-latex-classes (cons '("myarticle"
                                                        "% BEGIN My Article Defaults
    \\documentclass[10pt,letterpaper]{article}
    
    \\usepackage[letterpaper,includeheadfoot,top=0.5in,bottom=0.5in,left=0.75in,right=0.75in]{geometry}
    \\usepackage[utf8]{inputenc}
    \\usepackage[T1]{fontenc}
    \\usepackage{hyperref}
    \\usepackage{lastpage}
    \\usepackage{fancyhdr}
    \\usepackage[all]{xy}              %for doing graphics http://en.wikibooks.org/wiki/LaTeX/Creating_Graphics#Xy-pic
    \\pagestyle{fancy}
    \\usepackage{upquote}              %so verbatim code doesn't output as curly quotes
    \\renewcommand{\\headrulewidth}{1pt}
    \\renewcommand{\\footrulewidth}{0.5pt}
    
    % For the engineering phasor symbol
    % http://newsgroups.derkeiler.com/Archive/Comp/comp.text.tex/2006-02/msg00895.html
    \\makeatletter
    \\def\\phase#1{{\\vbox{\\ialign{$\\m@th\\scriptstyle##$\\crcr
    \\not\\mathrel{\\mkern10mu\\smash{#1}}\\crcr
    \\noalign{\\nointerlineskip}
    \\mkern2.5mu\\leaders\\hrule height.34pt\\hfill\\mkern2.5mu\\crcr}}}}
    \\makeatother
    
    % Default footer
    \\fancyfoot[L]{\\small \\jobname \\\\ \\today}
    \\fancyfoot[C]{\\small Page \\thepage\\ of \\pageref{LastPage}}
    \\fancyfoot[R]{\\small \\copyright \\the\\year\\  Chris Beard}
    % END My Article Defaults
    
    "
                                                        ("\\section{%s}" . "\\section*{%s}")
                                                        ("\\subsection{%s}" . "\\subsection*{%s}")
                                                        ("\\subsubsection{%s}" . "\\subsubsection*{%s}")
                                                        ("\\paragraph{%s}" . "\\paragraph*{%s}")
                                                        ("\\subparagraph{%s}" . "\\subparagraph*{%s}"))
                                                      org-export-latex-classes))))))

(add-to-list 'org-export-latex-classes
             '("myreport"
               "% BEGIN My Report Defaults
    \\documentclass[10pt,letterpaper]{article}
    
    \\usepackage[letterpaper,includeheadfoot,top=1in,bottom=.75in,left=.75in,right=.75in]{geometry}
    \\usepackage[utf8]{inputenc}
    \\usepackage[T1]{fontenc}
    \\usepackage{hyperref}
    \\usepackage{lastpage}
    \\usepackage{fancyhdr}
    %\\usepackage[all]{xy}              %for doing graphics http://en.wikibooks.org/wiki/LaTeX/Creating_Graphics#Xy-pic
    \\pagestyle{fancy}
    %\\usepackage{upquote}              %so verbatim code doesn't output as curly quotes
    \\renewcommand{\\headrulewidth}{1pt}
    \\renewcommand{\\footrulewidth}{0.5pt}
    
    % For the engineering phasor symbol
    % http://newsgroups.derkeiler.com/Archive/Comp/comp.text.tex/2006-02/msg00895.html
    \\makeatletter
    \\def\\phase#1{{\\vbox{\\ialign{$\\m@th\\scriptstyle##$\\crcr
    \\not\\mathrel{\\mkern10mu\\smash{#1}}\\crcr
    \\noalign{\\nointerlineskip}
    \\mkern2.5mu\\leaders\\hrule height.34pt\\hfill\\mkern2.5mu\\crcr}}}}
    \\makeatother
    
    % Default footer
    %\\fancyfoot[L]{\\small \\jobname \\\\ \\today}
    %\\fancyfoot[C]{\\small Page \\thepage\\ of \\pageref{LastPage}}
    %\\fancyfoot[R]{\\small \\copyright \\the\\year\\  Chris Beard}
    % END My Article Defaults
    
    "
               ("\\section{%s}" . "\\section*{%s}")
               ("\\subsection{%s}" . "\\subsection*{%s}")
               ("\\subsubsection{%s}" . "\\subsubsection*{%s}")
               ("\\paragraph{%s}" . "\\paragraph*{%s}")
               ("\\subparagraph{%s}" . "\\subparagraph*{%s}")))
\end{verbatim}
\subsection*{Org-Mode Hook}
\label{sec-1_2}

The latest version of yasnippets doesn't play well with Org-mode, the
following function allows these two to play nicely together
\begin{verbatim}
  (defun yas/org-very-safe-expand ()
    (let ((yas/fallback-behavior 'return-nil)) (yas/expand)))
  
  (add-hook 'org-mode-hook
            (lambda ()
              ;; yasnippet (using the new org-cycle hooks)
              (make-variable-buffer-local 'yas/trigger-key)
              (setq yas/trigger-key [tab])
              (add-to-list 'org-tab-first-hook 'yas/org-very-safe-expand)
              (define-key yas/keymap [tab] 'yas/next-field)))

;; (add-hook 'org-mode-hook
;;             (lambda ()
;;               (LaTeX-math-mode)))
\end{verbatim}

\begin{verbatim}
(add-hook 'org-mode-hook
          (lambda ()
            (local-set-key "\M-\C-n" 'outline-next-visible-heading)
            (local-set-key "\M-\C-p" 'outline-previous-visible-heading)
            (local-set-key "\M-\C-u" 'outline-up-heading)
            ;; table
            (local-set-key "\M-\C-w" 'org-table-copy-region)
            (local-set-key "\M-\C-y" 'org-table-paste-rectangle)
            (local-set-key "\M-\C-l" 'org-table-sort-lines)
            ;; display images
            (local-set-key "\M-I" 'org-toggle-iimage-in-org)
            ;; yasnippet (using the new org-cycle hooks)
            (make-variable-buffer-local 'yas/trigger-key)
            (setq yas/trigger-key [tab])
            (add-to-list 'org-tab-first-hook 'yas/org-very-safe-expand)
            (define-key yas/keymap [tab] 'yas/next-field)
            ))
\end{verbatim}
\subsection*{My org file settings}
\label{sec-1_3}

\begin{verbatim}
  ;; Not sure
(defvar org-sys-directory "~/devel/org-mode/lisp"
  "/opt/local/var/macports/sources/rsync.macports.org/release/ports/editors/org-mode 
     ")
(setq load-path (cons org-sys-directory load-path))

;; Set to the location of your Org files on your local system
(setq org-directory "~/Dropbox/org")
(setq org-mobile-directory "~/Dropbox/MobileOrg")
(setq org-mobile-inbox-for-pull "~/Dropbox/org/from-mobile.org")
;; (setq org-agenda-files (quote ("~/Dropbox/org/")))
(custom-set-variables '(org-agenda-files (quote ("~/Dropbox/org/"
                                                 ;; "~/Dropbox/org/school.org"
                                                 ;; "~/Dropbox/org/leisure.org"
                                                 ))))

(setq org-tag-alist '(("libertarian" . ?l)
                      ("mirism" . ?m)
                      ("emacs" . ?e)
                      ("thought" . ?i)
                      ("engineer" . ?n)
                      ("org-mode" . ?o)
                      ("gradapp" . ?g)
                      ))

(setq org-todo-keywords
      '((sequence "TODO(t)"  "|" "DONE(d)" "STARTED(s)""PARTIAL(p)" "WAITING(w)" )))

;; Capture
;; http://orgmode.org/manual/Template-expansion.html#Template-expansion
(setq org-default-notes-file (concat org-directory "/notes.org"))
(setq org-capture-templates
      '(("t" "Todo" entry (file+headline "~/Dropbox/org/notes.org" "Tasks")
         "* TODO %?\n %u  %i")
        ("j" "Journal" entry (file+datetree "~/Dropbox/org/thoughts-in-life3.org")
         "* %?\n %u  %i\n %a")
        ("i" "List Item" item (file "~/Dropbox/org/notes.org")
         "- %?\n %u  %i\n %a")
        ))


;; Shortcut Keys
(global-set-key "\C-cl" 'org-store-link)
(global-set-key "\C-ca" 'org-agenda)
(global-set-key "\C-cb" 'org-iswitchb)
(define-key global-map "\C-cc" 'org-capture)
\end{verbatim}
\subsection*{Open org files in org\ldots{}I think?}
\label{sec-1_4}

\begin{verbatim}
(add-to-list 'auto-mode-alist '("\\.org\\'" . org-mode))
\end{verbatim}
   
\subsection*{Speed keys}
\label{sec-1_5}

Speed commands enable single-letter commands in Org-mode files when
the point is at the beginning of a headline, or at the beginning of a
code block.

See the `=org-speed-commands-default=' variable for a list of the keys
and commands enabled at the beginning of headlines.  All code blocks
are available at the beginning of a code block, the following key
sequence \texttt{C-c C-v h} (bound to `=org-babel-describe-bindings=') will
display a list of the code blocks commands and their related keys.

\begin{verbatim}
(setq org-use-speed-commands t)

; ==============
; ===MY STUFF===
; ==============
(add-to-list 'org-speed-commands-user
             '("h" org-speed-move-safe (quote outline-next-visible-heading)))
(add-to-list 'org-speed-commands-user
             '("t" org-speed-move-safe (quote outline-previous-visible-heading)))
(add-to-list 'org-speed-commands-user
             '("d" org-speed-move-safe (quote org-backward-same-level)))
(add-to-list 'org-speed-commands-user
             '("n" org-speed-move-safe (quote org-forward-same-level)))
(add-to-list 'org-speed-commands-user
             '("g" org-speed-move-safe (quote outline-up-heading)))
\end{verbatim}
\subsection*{Code blocks}
\label{sec-1_6}

This activates a number of widely used languages, you are encouraged
to activate more languages using the customize interface for the
`=org-babel-load-languages=' variable, or with an elisp form like the
one below.  The customize interface of `=org-babel-load-languages='
contains an up to date list of the currently supported languages.
\begin{verbatim}
(org-babel-do-load-languages
 'org-babel-load-languages
 '((emacs-lisp . t)
   (sh . t)
   (R . t)
   (perl . t)
   (ruby . t)
   (python . t)
   (js . t)
   (haskell . t)
   (clojure . t)
   (ditaa . t)))
\end{verbatim}

You are encouraged to add the following to your personal configuration
although it is not added by default as a security precaution.
\begin{verbatim}
(setq org-confirm-babel-evaluate nil)
\end{verbatim}

The following displays the contents of code blocks in Org-mode files
using the major-mode of the code.  It also changes the behavior of
\texttt{TAB} to as if it were used in the appropriate major mode.  This means
that reading and editing code form inside of your Org-mode files is
much more like reading and editing of code using its major mode.
\begin{verbatim}
(setq org-src-fontify-natively t)
(setq org-src-tab-acts-natively t)
\end{verbatim}
\subsection*{Load up the Library of Babel}
\label{sec-1_7}

The library of babel contains makes many useful functions available
for use by code blocks in \textbf{any} emacs file.  See the actual
\href{file:///Users/FingerMan/.emacs.d/src/org/contrib/babel/library-of-babel.org}{library-of-babel.org} file for information on the functions, and see
\href{http://orgmode.org/worg/org-contrib/babel/intro.php#library-of-babel}{worg:library-of-babel} for more usage information.
\begin{verbatim}
(org-babel-lob-ingest
 (expand-file-name
  "library-of-babel.org"
  (expand-file-name
   "babel"
   (expand-file-name
    "contrib"
    (expand-file-name
     "org"
     (expand-file-name "src" dotfiles-dir))))))
\end{verbatim}
\subsection*{Ensure the Latest Org-mode manual is in the info directory}
\label{sec-1_8}

By placing the \texttt{doc/} directory in Org-mode at the front of the
\texttt{Info-directory-list} we can be sure that the latest version of the
Org-mode manual is available to the \texttt{info} command (bound to \texttt{C-h i}).
\begin{verbatim}
(if (boundp 'Info-directory-list)
    (setq Info-directory-list (append Info-directory-list
                                      Info-default-directory-list))
    (setq Info-directory-list Info-default-directory-list))
(setq Info-directory-list
      (cons (expand-file-name
             "doc"
             (expand-file-name
              "org"
              (expand-file-name "src" dotfiles-dir)))
            Info-directory-list))
\end{verbatim}
\subsection*{Starter Kit Documentation}
\label{sec-1_9}

This code defines the \texttt{starter-kit-project} which is used to publish
the documentation for the Starter Kit to html.

\begin{verbatim}
(setq org-export-htmlize-output-type 'css)
(unless (boundp 'org-publish-project-alist)
  (setq org-publish-project-alist nil))
(let ((this-dir (file-name-directory (or load-file-name buffer-file-name))))
  (add-to-list 'org-publish-project-alist
               `("starter-kit-documentation"
                 :base-directory ,this-dir
                 :base-extension "org"
                 :style "<link rel=\"stylesheet\" href=\"emacs.css\" type=\"text/css\"/>"
                 :publishing-directory ,this-dir
                 :index-filename "starter-kit.org"
                 :auto-postamble nil
                 :postamble nil)))
\end{verbatim}

\end{document}
